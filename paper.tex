% $Id$
\documentclass{ansarticle}

\title{Preparing articles for submission to the \\
       Archive of Numerical Software}
\author[1]{Wolfgang Bangerth}
\author[2]{Peter Bastian}
\author[1]{Guido Kanschat\thanks{Additional thanks to my neighbor's dog
  for waking me up on time to work on this style file}}
\author[3]{Michael Heroux}
\author[4]{Anders Logg}
\affil[1]{Department of Mathematics, Texas A\&M University}
\affil[2]{IWR, Universit\"at Heidelberg}
\affil[3]{Sandia National Laboratories}
\affil[4]{Simula Research Laboratory}
\runningtitle{Preparing articles for ANS}
\runningauthor{Bangerth, Bastian, Kanschat, Heroux, Logg}

%------------------------------------------------------------------------------
\begin{document}

\maketitle

\begin{abstract}
  This document describes the general structure of the
  \texttt{ansarticle} document class and serves as an example for its
  usage. \texttt{ansarticle.cls} is based on the standard \LaTeX{}
  \texttt{article.cls} and a few other standard packages. It fixes
  style parameters to the preferred style of the Archive of Numerical
  Software (ANS).
\end{abstract}

%------------------------------------------------------------------------------
\section{Requirements}

To write an article for ANS, you need to download the following files
from the ANS web page:
\begin{itemize}
\item
  \emp{paper.tex}, the source file for this document and a template for
  authors;
\item
  \emp{ansarticle.cls}, the ANS document class file;
\item
  \emp{anslistings.sty}, the ANS style file for typesetting of code;
\item
  \emp{bibliography.bib}, a bibliography with some important
  references;
\item
  \emp{Makefile}, useful for building your document.
\end{itemize}

The file \texttt{ansarticle.cls} loads the standard LaTeX
\texttt{article.cls} and some additional packages. It also sets up the
layout. Hardly any of the options of the packages are available to the
author, in order to keep the journal style consistent.

Upon loading, \texttt{ansarticle.cls} requires the following packages:
\emp{a4wide},
\emp{anslistings},
\emp{algorithm},
\emp{algorithmicx},
\emp{amsmath},
\emp{amssymb},
\emp{booktabs},
\emp{fancyhdr},
\emp{graphicx},
\emp{hyperref},
\emp{authblk},
\emp{lastpage},
\emp{listings.}
These packages are part of standard \TeX{} distributions and should
not cause any difficulties.

%------------------------------------------------------------------------------
\section{Usage}

\subsection{Building the document}

Use this file (\emp{paper.tex}) as a template for your article. Then
simply type \emp{make} to build your article. This will call
\emp{pdflatex} to generate the PDF file \emp{paper.pdf}. To also call
\emp{bibtex}, type \emp{make final}.

\subsection{Typesetting mathematical formul\ae}

The ANS document class uses the amsmath and amssymb packages for
improved typesetting of mathematical formul\ae like, e.g., this one:
\begin{equation}
  \cfrac{1}{\sqrt{2}+
    \cfrac{1}{\sqrt{2}+
      \cfrac{1}{\sqrt{2}+\dotsb
  }}}.
\end{equation}

Make sure that you familiarize yourself with the \emp{amsmath}
environments \emp{multline}, \emp{align} and \emp{split} (and know the
differences between them). Avoid home-made splitting using
\emp{\{array\}\{rcl\}}.

\subsection{Typesetting tables}

\fixme{Write something about typesetting tables with \emp{booktabs}.}

\subsection{Typesetting code}

All code listings are based on the \texttt{listings}
package~\cite{HeinzMoses07}. The style file \texttt{anslistings.sty},
automatically included by \texttt{ansarticle.cls}, provides standard
styles for a variety of common programming languages. Changing any
listing style parameters is not encouraged and may lead to unexpected
outcomes in the final publication.

Below, we provide samples of listings for the languages available. If
a language is missing, please contact the managing editor for an
updated style file.

\subsubsection{C++}

Code should be compliant with the current C++ standard. Two commands
are provided to prettyprint C++ code. First, an environment to put C++
code into the LaTeX file, namely

\fixme{Adjust width of code box, a little too wide now.}

\begin{latexcode}
\begin{c++}
  // Your C++ code here
\end{c++}
\end{latexcode}

Alternatively, it is possible to use
\lstinline[language=TeX]!\inputcpp{file.cc}!  to print the C++ code in
a file. Here is an example what C++ code looks like.

\begin{c++}
// Get dimensions of local mesh_data
const unsigned int num_local_cells = mesh_data.cell_vertices.size();
assert(global_cell_indices.size() == num_local_cells);
const unsigned int num_cell_vertices = mesh_data.cell_vertices[0].size();

// Build array of cell-vertex connectivity and partition vector
std::vector<unsigned int> cell_vertices;
std::vector<unsigned int> cell_vertices_partition;
const unsigned int size = num_local_cells*(num_cell_vertices + 1);
cell_vertices.reserve(size);
cell_vertices_partition.reserve(size);
for (unsigned int i = 0; i < num_local_cells; i++)
{
  cell_vertices.push_back(global_cell_indices[i]);
  cell_vertices_partition.push_back(cell_partition[i]);
  for (unsigned int j = 0; j < num_cell_vertices; j++)
  {
    cell_vertices.push_back(mesh_data.cell_vertices[i][j]);
    cell_vertices_partition.push_back(cell_partition[i]);
  }
}

// Distribute cell-vertex connectivity
MPI::distribute(cell_vertices, cell_vertices_partition);
assert(cell_vertices.size());
cell_vertices_partition.clear();
\end{c++}

\subsubsection{Python}
can be typeset as well.

\begin{python}
# Time-stepping
t = dt
p = Progress("Time-stepping")
while t < T + DOLFIN_EPS:

    # Update pressure boundary condition
    p_in.t = t

    # Compute tentative velocity step
    begin("Computing tentative velocity")
    b1 = assemble(L1)
    [bc.apply(A1, b1) for bc in bcu]
    solve(A1, u1.vector(), b1, "gmres", "ilu")
    end()

    # Pressure correction
    begin("Computing pressure correction")
    b2 = assemble(L2)
    [bc.apply(A2, b2) for bc in bcp]
    solve(A2, p1.vector(), b2, "gmres", "amg_hypre")
    end()

    # Velocity correction
    begin("Computing velocity correction")
    b3 = assemble(L3)
    [bc.apply(A3, b3) for bc in bcu]
    solve(A3, u1.vector(), b3, "gmres", "ilu")
    end()

    # Plot solution
    plot(p1, title="Pressure", rescale=True)
    plot(u1, title="Velocity", rescale=True)

    # Save to file
    ufile << u1
    pfile << p1

    # Move to next time step
    u0.assign(u1)
    p.update(t / T)
    t += dt
\end{python}

\subsection{Handling supplementary material}

Additional material, chich is not part of the article text, but part
of the submission, for instance setup scripts or video output, can be
linked from the article, using the \texttt{hyperref} commands
(\texttt{hyperref} is automaticly included by
\texttt{ansarticle}). URLs to such material should start with the
filename. A suitable prefix will be added to the links in the
editorial process.

%------------------------------------------------------------------------------
\section{Production after article was accepted}
\subsection{Publication metadata}

Create a file \texttt{ansinfo.tex} in the same directory. The file
contains macros defining the publication information like in the
following listing:

\inputlatex{ansinfo.tex}

%------------------------------------------------------------------------------
\section{Philosophi\ae{} Naturalis Principia Mathematica}

Excerpt from~\cite{Newton1687}.

\subsection{Pr\ae{}fatio ad Lectorem}

Cum Veteres Mechanicam (uti Author est Pappus) in verum Naturalium
investigatione maximi fecerint, \& recentiores, missis formis
substantialibus \& qualitatibus occultis, Ph\ae{}nomena Natur\ae{} ad
leges Mathematicas revocare aggressi sint: Visum est in hoc Tractatu
Mathesin excolere quatenus ea ad Philosophiam spectat. Mechanicam vero
duplicem Veteres constituerunt: Rationalem qu\ae{} per Demonstrationes
accurate procedit, \& Practicam. Ad practicam spectant Artes omnes
Manuales, a quibus utiq; Mechanica nomen mutuata est. Cum autem
Artifices parum accurate operari soleant, fit ut Mechanica omnis a
Geometria ita distinguatur, ut quicquid accuratum sit ad Geometriam
referatur, quicquid minus accuratum ad Mechanicam. Attamen errores non
sunt Artis sed Artificum. Qui minus accurate operatur, imperfectior
est Mechanicus, \& si quis accuratissime operari posset, hic foret
Mechanicus omnium perfectissimus. Nam \& Linearum rectarum \&
Circulorum descriptiones in quibus Geometria fundatur, ad Mechanicam
pertinent. Has lineas describere Geometria non docet sed
postulat. Postulat enim ut Tyro easdem accurate describere prius
didicerit quam limen attingat Geometri\ae{}; dein, quomodo per has
operationes Problemata solvantur, docet. Rectas \& circulos describere
Problemata sunt sed non Geometrica. Ex Mechanica postulatur horum
solutio, in Geometria docetur solutorum usus. Ac gloriatur Geometria
quod tam paucis principiis aliunde petitis tam multa
pr\ae{}stet. Fundatur igitur Geometria in praxi Mechanica, \& nihil
aliud est quam Mechanic\ae{} universalis pars illa qu\ae{} artem
mensurandi accurate proponit ac demonstrat. Cum autem artes Manuales
in corporibus movendis pr\ae{}cipue versentur, fit ut Geometria ad
magnitudinem, Mechanica ad motum vulgo reseratur. Quo sensu Mechanica
rationalis erit Scientia Motuum qui ex viribus quibuscunq; resultant,
\& virium qu\ae{} ad motus quoscunq; requiruntur, accurate proposita
ac demonstrata. Pars h\ae{}c Mechanic\ae{} a Veteribus in Potentiis
quinque ad artes manuales spectantibus exculta fuit, qui Gravitatem
(cum potentia manualis non sit) vix aliter quam in ponderibus per
potentias illas movendis considerarunt. Nos autem non Artibus sed
Philosophi\ae{} consulentes, deq; potentiis non manualibus sed
naturalibus scribentes, ea maxime tractamus qu\ae{} ad Gravitatem,
levitatem, vim Elasticam, resistentiam Fluidorum \& ejusmodi vires seu
attractivas seu impulsivas spectant: Et ea propter h\ae{}c nostra
tanquam Philosophi\ae{} principia Mathematica proponimus. Omnis enim
Philosophi\ae{} difficultas in eo versari videtur, ut a Ph\ae{}nomenis
motuum investigemus vires Natur\ae{}, deinde ab his viribus
demonstremus ph\ae{}nomena reliqua. Et hac spectant Propositiones
generales quas Libro primo \& secundo pertractavimus. In Libro autem
tertio exemplum hujus rei proposuimus per explicationem Systematis
mundani. Ibi enim, ex ph\ae{}nomenis c\ae{}lestibus, per Propositiones
in Libris prioribus Mathematice demonstratas, derivantur vires
gravitatis quibus corpora ad Solem \& Planetas singulos
tendunt. Deinde ex his viribus per Propositiones etiam Mathematicas
deducuntur motus Planetarum, Cometarum, Lun\ae{} \& Maris. Utinam
c\ae{}tera Natur\ae{} ph\ae{}nomena ex principiis Mechanicis eodem
argumentandi genere derivare liceret. Nam multa me movent ut nonnihil
suspicer ea omnia ex viribus quibusdam pendere posse, quibus corporum
particul\ae{} per causas nondum cognitas vel in se mutuo impelluntur
\& secundum figuras regulares coh\ae{}rent, vel ab invicem fugantur \&
recedunt: quibus viribus ignotis, Philosophi hactenus Naturam frustra
tentarunt. Spero autem quod vel huic Philosophandi modo, vel veriori
alicui, Principia hic posita lucem aliquam pr\ae{}bebunt.

In his edendis, Vir acutissimus \& in omni literarum genere
eruditissimus Edmundus Halleius operam navavit, nec solum Typothetarum
Sphalmata correxit \& Schemata incidi curavit, sed etiam Author fuit ut
horum editionem aggrederer. Quippe cum demonstratam a me figuram
Orbium c\ae{}lestium impetraverat, rogare non destitit ut eadem cum
Societate Regali communicarem, Qu\ae{} deinde hortatibus \& benignis suis
auspiciis effecit ut de eadem in lucem emittenda cogitare
inciperem. At postquam Motuum Lunarium in\ae{}qualitates aggressus essem,
deinde etiam alia tentare c\ae{}pissem qu\ae{} ad leges mensuras Gravitatis \&
aliarum virium, ad figuras a corporibus secundum datas quascunque
leges attractis describendas, ad motus corporum plurium inter se, ad
motus corporum in Mediis resistentibus, ad vires, densitates \& motus
Mediorum, ad Orbes Cometarum \& similia spectant, editionem in aliud
tempus differendam esse putavi, ut c\ae{}tera rimarer \& una in publicum
darem. Qu\ae{} ad motus Lunares spectant, (imperfecta cum sint,) in
Corollariis Propositionis LXVI. simul complexus sum, ne singula
methodo prolixiore quam pro rei dignitate proponere, \& sigillatim
demonstrare tenerer, \& seriem reliquarum Propositionum
interrumpere. Nonnulla sero inventa locis minus idoneis inserere
malui, quam numerum Propositionum \& citationes mutare. Ut omnia
candide legantur, \& defectus, in materia tam difficili non tam
reprehendantur, quam novis Lectorum conatibus investigentur, \& benigne
suppleantur, enixe rogo.

\subsection{Definitiones}

\subsubsection{Def. I}

Quantitas Materi\ae{} est mensura ejusdem orta ex illius Densitate \&
Magnitudine conjunctim.

\paragraph{Aer}  duplo densior in duplo spatio quadruplus est. Idem
intellige de Nive et Pulveribus per compressionem vel liquefactionem
condensatis. Et par est ratio corporum omnium, qu\ae{} per causas
quascunq; diversimode condensantur. Medii interea, si quod fuerit,
interstitia partium libere pervadentis, hic nullam rationem
habeo. Hanc autem quantitatem sub nomine corporis vel Mass\ae{} in
sequentibus passim intelligo. Innotescit ea per corporis cujusq;
pondus. Nam ponderi proportionalem esse reperi per experimenta
pendulorum accuratissime instituta, uti posthac docebitur.

\subsubsection{Def. II}
Quantitas motus est mensura ejusdem orta ex Velocitate et quantitate
Materi\ae{} conjunctim.

\paragraph{Motus}
totius est summa motuum in partibus singulis, adeoq; in corpore duplo
majore \ae{}quali cum Velocitate duplus est, et dupla cum Velocitate
quadruplus.

\subsubsection{Def. III}
Materi\ae{} vis insita est potentia resistendi, qua corpus unumquodq;,
quantum in se est, perseverat in statu suo vel quiescendi vel movendi
uniformiter in directum.

\paragraph{H\ae{}c}
semper proportionalis est suo corpori, neq; differt quicquam ab
inertia Mass\ae{}, nisi in modo concipiendi. Per inertiam materi\ae{} fit ut
corpus omne de statu suo vel quiescendi vel movendi difficulter
deturbetur. Unde etiam vis insita nomine significantissimo vis inerti\ae{}
dici possit. Exercet vero corpus hanc vim solummodo in mutatione
status sui per vim aliam in se impressam facta, estq; exercitium ejus
sub diverso respectu et Resistentia et Impetus: Resistentia quatenus
corpus ad conservandum statum suum reluctatur vi impress\ae{}; Impetus
quatenus corpus idem, vi resistentis obstaculi difficulter cedendo,
conatur statum ejus mutare. Vulgus Resistentiam quiescentibus et
Impetum moventibus tribuit; sed motus et quies, uti vulgo
concipiuntur, respectu solo distinguuntur ab invicem, neq; semper vere
quiescunt qu\ae{} vulgo tanquam quiescentia spectantur.

\subparagraph{A subparagraph} If you really want to nest that deeply, go
ahead! But, while you might have enjoyed
reading~\cite{Wittgenstein21,Wittgenstein81}, you are encouraged to
question the applicability to a publication on a thing as profane as
software.

\bibliographystyle{abbrv}
\bibliography{submitting}
\end{document}

%%% Local Variables:
%%% mode: latex
%%% TeX-master: t
%%% End:
